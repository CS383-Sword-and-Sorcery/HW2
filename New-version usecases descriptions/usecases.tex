\documentclass[12pt,letterpaper]{scrreprt}

%----------------------------------------------------------------------
%				Required Packages
%----------------------------------------------------------------------
\usepackage{usecases}
\usepackage{enumitem}
%\usepackage{paralist}
%\usepackage{tabto}
\usepackage[top=2cm,bottom=3.5cm]{geometry}

%----------------------------------------------------------------------
%				Title Page
%----------------------------------------------------------------------
\title{CS383: Software Engineering}
\subtitle{HW2: Use Cases\\Spring 2014}
\author{Tao, John, Cameron, Gabe, Wayne, Joseph} % Include last names?
\date{}

%----------------------------------------------------------------------
%				Additional Settings
%----------------------------------------------------------------------
\setcounter{tocdepth}{3}
\setcounter{secnumdepth}{3}




%----------------------------------------------------------------------
%				Playtest Document Starts
%----------------------------------------------------------------------
\begin{document}

% Initializations
%\NumTabs{2} % This is for creating a two column use case scenario
\maketitle
\tableofcontents % Eventually uncomment this


%----------------------------------------------------------------------
%				Use Case Document Starts
%----------------------------------------------------------------------
\chapter{Use Cases}

%==================================================
%				Magic: Tao Zhang
%==================================================
\section{Magic}
\paragraph{(Author: Tao Zhang)}
            \subsection{Cast a Spell}
                \begin{usecase}
                  \addtitle{Magic I}{Cast a Spell}
                  \additemizedfield{Actors}{
                    \item Phasing player
                  }
                  \addfield{Goal}{Phasing player cast spells}
                  \additemizedfield{Preconditions}{
                    \item Phasing player is in one of the following phase
                    	\begin{enumerate}
                    		\item Movement
                    		\item Spell Segment
                    		\item Combat
                    	\end{enumerate}
                  }
                  \addfield{Summary}{Phasing player gonna cast enough spells he want during his turn}
                  \additemizedfield{Related Usecases}{
                  	\item Movement
                  	\item Combat                  
                  }
                  \addscenario{Steps}{
                  	\item Phasing Player select a character with PL
					\item Phasing player select an available spell
					\item player selects a target for the spell
					\item Player imform to cast this spell
					\item repeat to cast enough spells
                  }
                  \additemizedfield{Alternative}{
					\item Only in Movement and spell segment:
						\begin{enumerate}
							\item Phasing player can choose a higher level spell with warning red background.
							\item Computer will computes the result if successfully cast the spell and whether the charater survive or not
						\end{enumerate}	
					\item Player click on the buttom "End of casting spells" which the bottom will always be displayed on side of the screen					               
                  }
                \end{usecase}
                
                \subsection{Cast a CounterSpell}
                \begin{usecase}
                  \addtitle{Magic II}{Cast a CounterSpell}
                  \additemizedfield{Actors}{
                  	\item Non-phasing Player
                  }
                  \addfield{Goal}{Non-phasing players cast counterspells}
                  \additemizedfield{Preconditions}{
                  	\item End of phasing player's spell segment
                  }
                  \addfield{Summary}{All non-phasing player will do this at the same time, and once all of them have clicked the "end of counterspell" button. The server will then turn to let phasing player control}
                  \addscenario{Steps}{
                  	\item Non-phasing player select a charater with PL
					\item Select an available counterspell
					\item Select a target to cast this spell
					\item Repeat to select enough counterspells
                  }
                  \additemizedfield{Alternative}{
                  	\item Non-phasing player click the button "End of casting counterspells". 
                  }
                  
                \end{usecase}
                
                
%========================================================
%	Selection Section - Gabe Pearhill
%========================================================                         
\section{Selection}
\paragraph{(Author: Gabe Pearhill)}
            \subsection{Unit Selection}
                \begin{usecase}
                  \addtitle{Select Unit(s)}{Select one or more units}
                  \addfield{Summary}{Player clicks a unit on the game board.}
                  \additemizedfield{Actors}{
                    \item Player
                  }
                  \additemizedfield{Preconditions}{
                    \item Phase requiring unit selection.
                  }
                  \addscenario{Steps}{
                    \item Once a phase requiring unit selection begins, the computer highlights all available units. 
                    \item The user clicks one or more units.
                    \item Computer saves the selection state.
                  }
                  \end{usecase}
                
                 
            \subsection{Hexagon Selection}
                \begin{usecase}
                  \addtitle{Select Hexagon}{Record the players hexagon selection.}
                  \addfield{Summary}{The basic action of selecting a hexagon, be it for magic, movement, or attacking.}
                  \additemizedfield{Actors}{
                        \item Player
                  }
                  \addscenario{Steps}{
                                    \item Player clicks on a hex.
                                    \item Computer records the hex selection.
                  }
                \end{usecase}

%========================================
%	Movement Section - Gabe Pearhill
%========================================
\section{Movement}
\paragraph{(Author: Gabe Pearhill)}
            \subsection{Move a Unit}
                \begin{usecase}
                  \addtitle{Move Unit(s)}{Move unit(s) across the map!}
                  \addfield{Summary}{During the movement phase the player selects and moves units.}
                  \additemizedfield{Actors}{
                    \item Player
                  }
                  \additemizedfield{Preconditions}{
                    \item Movement Phase
                  }
                  \addscenario{Steps}{
                    \item Select unit(s). (See Unit Selection)
                    \item Computer highlights hexagons within range of the selected units.
                    \item Player selects an eligible hexagon.
                    \item Computer checks if tile has special attributes (a portal for example) and takes action appropriately.
                    \item Handle items relating to zone of control.
                  }
                  \end{usecase}
                
                 
            \subsection{Using a Portal}
                \begin{usecase}
                  \addtitle{Teleportation}{Give the player the choice to use a portal hexagon.}
                  \addfield{Summary}{If a unit moves on top of a portal, and the player chooses to use it, the computer must move the selected units to another portal location on the map.}
                  \additemizedfield{Actors}{
                        \item Player
                  }
                  \addscenario{Steps}{
                                    \item Player moves on top of a portal hexagon.
                                    \item Player is provided a dialog giving them the option to use the portal.
                                    \item If the player chooses to use the portal the player must then choose to teleport his units individually or as a group.
                                    \item Perform appropriate teleportation.
                                    \item Should an enemy unit occupy an output portal, the teleported  units should be retreated by one tile.
                  }
                \end{usecase}

                 
\end{document}
